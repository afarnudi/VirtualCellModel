\subsubsection*{Line Width}

Try to use lines of code between 80 and 120 characters long.

\subsubsection*{Indentation}

We use 4 spaces for indentation.

\subsubsection*{Braces}


\begin{DoxyEnumerate}
\item In G\+NU style, if either side of an if-\/else statement has braces, both sides should, to match up indentation\+: 
\begin{DoxyCode}{0}
\DoxyCodeLine{\textcolor{keywordflow}{if} (condition):}
\DoxyCodeLine{    \{}
\DoxyCodeLine{        Foo;}
\DoxyCodeLine{        Bar;}
\DoxyCodeLine{    \}}
\DoxyCodeLine{\textcolor{keywordflow}{else}:}
\DoxyCodeLine{    \{}
\DoxyCodeLine{        Bar;}
\DoxyCodeLine{    \}}
\end{DoxyCode}

\item If the single statement covers multiple lines, e.\+g. for functions with many arguments, and it is followed by else or else if\+: 
\begin{DoxyCode}{0}
\DoxyCodeLine{\textcolor{keywordflow}{if} (condition):}
\DoxyCodeLine{    \{}
\DoxyCodeLine{        a\_single\_statement\_with\_many\_arguments (some\_lengthy\_argument,}
\DoxyCodeLine{                            another\_lengthy\_argument,}
\DoxyCodeLine{                            plus\_one);}
\DoxyCodeLine{    \}}
\DoxyCodeLine{\textcolor{keywordflow}{else}:}
\DoxyCodeLine{    \{}
\DoxyCodeLine{        another\_single\_statement (arg1, arg2);}
\DoxyCodeLine{    \}}
\end{DoxyCode}

\item If the condition is composed of many lines\+: 
\begin{DoxyCode}{0}
\DoxyCodeLine{\textcolor{keywordflow}{if}  (condition1 ||}
\DoxyCodeLine{    (condition2 \&\& condition3) ||}
\DoxyCodeLine{    condition4 ||}
\DoxyCodeLine{    (condition5 \&\& (condition6 || condition7))):}
\DoxyCodeLine{    \{}
\DoxyCodeLine{        A\_simple\_statement}
\DoxyCodeLine{    \}}
\end{DoxyCode}

\end{DoxyEnumerate}

\subsubsection*{Conditions}

Do not check boolean values for equality. By using implicit comparisons, the resulting code can be read more like conversational English. Another rationale is that a \textquotesingle{}true\textquotesingle{} value may not be necessarily equal to whatever the {\ttfamily T\+R\+UE} macro uses. The C language uses the value 0 for many purposes. As a numeric value, the end of a string, a null pointer and the F\+A\+L\+SE boolean. To make the code clearer, you should write code that highlights the speci?c way 0 is used. So when reading a comparison, it is possible to know the variable type. For boolean variables, an implicit comparison is appropriate because it\textquotesingle{}s already a logical expression.

\subsubsection*{Functions}

The argument list must be broken into a new line for each argument, with the argument names right aligned, taking into account pointers\+: 
\begin{DoxyCode}{0}
\DoxyCodeLine{\textcolor{keywordtype}{void}}
\DoxyCodeLine{My\_function (some\_type\_t    type,}
\DoxyCodeLine{         Another\_type\_t    *a\_pointer,}
\DoxyCodeLine{         Double\_ptr\_t     **double\_pointer)}
\DoxyCodeLine{\{}
\DoxyCodeLine{    ...}
\DoxyCodeLine{\}}
\end{DoxyCode}


The alignment also holds when invoking a function without breaking the line length limit.

\subsubsection*{Whitespace}

When declaring a structure type use newlines to separate logical sections of the structure\+: 
\begin{DoxyCode}{0}
\DoxyCodeLine{\textcolor{keyword}{struct }\_GtkWrapBoxPrivate}
\DoxyCodeLine{\{}
\DoxyCodeLine{    GtkOrientation        orientation;}
\DoxyCodeLine{    GtkWrapAllocationMode mode;}
\DoxyCodeLine{    }
\DoxyCodeLine{    GtkWrapBoxSpreading   horizontal\_spreading;}
\DoxyCodeLine{    GtkWrapBoxSpreading   vertical\_spreading;}
\DoxyCodeLine{\}}
\end{DoxyCode}


\subsubsection*{The switch Statement}

A switch should open a block on a new indentation level, and each case should start on the same indentation level as the curly braces, with the case block on a new indentation level\+: 
\begin{DoxyCode}{0}
\DoxyCodeLine{\textcolor{keywordflow}{switch} (condition) \{}
\DoxyCodeLine{\textcolor{keywordflow}{case} FOO:}
\DoxyCodeLine{    do\_foo ();}
\DoxyCodeLine{    \textcolor{keywordflow}{break};}
\DoxyCodeLine{\textcolor{keywordflow}{case} BAR:}
\DoxyCodeLine{    do\_bar;}
\DoxyCodeLine{    \textcolor{keywordflow}{break};}
\DoxyCodeLine{\}}
\end{DoxyCode}


\subsubsection*{Header Files}

The only major rule for headers is that the function definitions should be vertically aligned in three columns\+:


\begin{DoxyCode}{0}
\DoxyCodeLine{return\_type function\_name   (type   argument,}
\DoxyCodeLine{                                 type   argument,}
\DoxyCodeLine{                                 type   argument);}
\end{DoxyCode}


The maximum width of each column is given by the longest element in the column\+: 
\begin{DoxyCode}{0}
\DoxyCodeLine{\textcolor{keywordtype}{void}          gtk\_type\_set\_property (GtkType      *type,}
\DoxyCodeLine{                     \textcolor{keyword}{const} gchar  *value,}
\DoxyCodeLine{                     GError      **error);}
\DoxyCodeLine{\textcolor{keyword}{const} gchar  *gtk\_type\_get\_property (GtkType      *type);}
\end{DoxyCode}
 